\documentclass[11pt,dvipdfmx,b5paper]{jsbook}
%\documentclass[11pt,dvipdfmx,a5paper,oneside]{jsbook}

% -------------
% 基本パッケージ
% -------------
\usepackage{graphicx}
\usepackage{booktabs}
\usepackage{caption} % 図表のキャプションを変更する
\usepackage{geometry} % ページレイアウトを変更する
\usepackage{here} % 画像を強制的にその場に表示させる
\usepackage{framed} % 改ページ可能なフレームの生成
\usepackage{wrapfig} % 図の周りにテキストを回り込ませる
\usepackage{algorithm} % 疑似コードを書くためのパッケージ
\usepackage{listings} % ソースコードを書くためのパッケージ
%\usepackage{balance} % 2カラム文書を書く場合に左右カラムでバランスを取る

% -------------
% フォント系パッケージ
% -------------
\usepackage[utf8]{inputenc} % 文字コードを指定
\usepackage{microtype}
\usepackage[T1]{fontenc} % フォントのエンコーディングを指定
\usepackage{lmodern} % ラテンモダンフォントを指定
\usepackage{xcolor} % 文字の色を変えられるように

% -------------
% 数学表記系パッケージ
% -------------
\usepackage{latexsym} % 数式で使える記号を増やす
\usepackage{amsmath} % 数式を書くためのパッケージ
\usepackage{bm} % 太字のベクトルを書けるように
\usepackage{amsfonts}

% -------------
% リンク/目次系パッケージ
% -------------
\usepackage{url} % URLを書けるように
\usepackage[hidelinks]{hyperref} % ハイパーリンク付き文書を書けるように
\usepackage{pxjahyper} % hyperrefの機能を拡張する

% -------------
% 装飾系パッケージ
% -------------
%\usepackage{quotchap} % 章見出しを装飾する
%\usepackage{titlesec} 見出しの書式を変える
\usepackage[most]{tcolorbox} % 枠囲みを作成するパッケージ
%\usepackage{tikz} % tcolorboxとセットで使う

% -------------
% 執筆補助パッケージ
% -------------
\usepackage{lineno} % 行番号を表示
%\linenumbers

% -----------------
% 自作の関数・スタイル
% -----------------
% -------------
% 文字数・行数設定
% -------------
% 1行あたりの文字数
\setlength { \textwidth } { 37zw }
% 1ページあたりの行数
\setlength { \textheight } { 32\baselineskip }

% -------------
% 自作カラー
% -------------
\definecolor{brandblue}{rgb}{0.34, 0.7, 1}
\definecolor{background}{rgb}{0.94,0.95,0.96}
\definecolor{turquoisegreen}{rgb}{0.63, 0.84, 0.71}

% -------------
% 自作修飾
% -------------
%\newcommand{\red}[1]{\textcolor{red}{#1}}
\newcommand{\graybox}[1]{\colorbox[gray]{0.9}{\texttt{#1}}}
\newcommand{\strong}[1]{\textbf{#1}}

% -------------
% 自作スタイル
% -------------
\tcbset {
  base/.style={
    arc=0mm,
    bottomtitle=0.5mm,
    boxrule=0mm,
    %colbacktitle=black!10!white,
    coltitle=black,
    fonttitle=\bfseries,
    %leftrule=1mm,
    %left=2.5mm,
    left=3.5mm,
    right=3.5mm,
    title={#1},
    toptitle=0.75mm,
  }
}

\newtcolorbox{mainbox}[1]{
  colframe=brandblue,
  base={#1}
}

\newtcolorbox{notebox}[1]{
  colframe=brandblue,
  base={Note - #1}
}

\newtcolorbox{warningbox}[1]{
  colframe=orange,
  base={注意 - #1}
}

\newtcolorbox{tipbox}[1]{
  colframe=turquoisegreen,
  base={Tip - #1}
}

\newtcolorbox{subbox}[1]{
  colframe=black!30!white,
  base={#1}
}

\usepackage{minted}
\usemintedstyle[sql]{tango}


% ------------
% コンテンツ
% ------------
% 森北出版さんからの依頼事項
% - ページ数についてはざっくり「200ページ程度」
% - これにはまえがき等も含まれる
% - 1ページあたりの文字数は37文字×32行
% ------------
\begin{document}
% 前書き
\frontmatter
\section*{前書き}
データベースという言葉を知ったのは,大学3年生の授業であった.
当時,工学部情報科学科の学生であった私は,必修科目である「データベース」の講義を受けていたが,他の授業と同様,ただ漫然と講義を受けていた.
単位を修得をしたものの,データベースの意義を理解していなかった.

その後,縁があってデータベース研究者として名を馳せた先生の研究室に入り,情報検索の研究活動を行っていたが...
まさかデータベースの教科書を自分が書くとは思ってもいなかった.


% 目次
\tableofcontents

% コンテンツ
\mainmatter
\chapter{データベースを使わない世界}

データベースと聞いて何を思い浮かべるだろうか.
大抵の人は,データベースといえば「大きなデータの集まり」といったイメージを持つのではないだろうか.
このイメージはあながち間違いではない.

さて,本講義のようにわざわざ科目を立ててまで,データベースについて学ぶことはあるのだろうか?
結論としては,大規模データに携わるITエンジニアやデータ分析者を目指す人であれば,「大いにあり」である.
1960年代から今日に至るまで,データベース技術は盛んに研究開発が行われてきた.
大きなデータの集まりを扱うには,対処しなければならない問題が思った以上に数多く存在するのである.

本講義では10数回にわたってデータベース技術について解説するが,この第1講ではデータベースのことはいったん横に置いておいておく.
今回は(割と)大きなデータを扱うときに遭遇する問題について考えてみよう.


\section{ケース1: 販売履歴の記録をはじめる}
以下は,山畑さんという架空の人物のお話である.

\begin{framed}
山畑さんは家族で小さな小売店を営んでいる.
個人経営ながら山畑さんのお店は繁盛している.
とはいえ,街には大手チェーン小売店が進出してきており,このまま順調に経営を続けられるか,不安が募っている.
何か手を打たなければならない.

2020年の4月,山畑さんは念願のショッピングサイトを立ち上げた.
言うまでもない.
ショッピングサイトを立ち上げたのは,オンラインの場にも顧客獲得の機会を求めるためだ.
サイトは順調に立ち上がり,注文もポツポツ入ってきている.

ところで,最近「データサイエンス」なるものが世間の注目を集めているらしい.
データを活かせばビジネスチャンスが広がるとのことだ.
山畑さんは,Excelシートに記録を取り始めた販売履歴を分析してみようと思い立った.

山畑さんが使っているExcelシートには,「いつ,誰が,何を,いくらで購入したか」の情報が記録されている.
ショッピングサイトは立ち上がったばかりであり,Excelシートには200行しかデータが入っていない.
しかし,今後データが貯まっていけば,売り上げを増やすための課題が見えるかもしれない.
いずれがっつりとデータ分析をやるためにも,山畑さんは手持ちのデータを用いて分析の練習に取り組むことにした.
\end{framed}


\subsubsection{Q1. データの確認}
こちらのURL(\url{https://dbnote.hontolab.org/data/purchase\_small.xlsx})から,上のケース1で山畑さんがデータ分析の練習に使おうとしているExcelファイル(\graybox{purchase\_small.xlsx})をダウンロードし,中身を確認しなさい.

なお,Excelシートの各列の意味は以下の通り:
\begin{itemize}
\item purchased\_at: 購買(販売)日時
\item customer: 商品を購入した人物の氏名
\item gender: 商品を購入した人物の性別
\item product: 購入された商品名
\item sale: 販売価格
\end{itemize}


\subsubsection{Q2. あの人は何回買い物をしている?}
「岡田 真綾」という人物が何回買い物をしていたかを数えなさい.


\subsubsection{Q3. 商品Xを購入しているのは誰?}
Excelのオートフィルタ機能を使って,「ビタミン補助剤」を購入している人をリストアップしなさい.


\subsubsection{Q4. 総売上金額}
Excel関数の\graybox{\texttt{SUM}}を用いて,現時点での総売上金額を計算しなさい.


\subsubsection{Q5. 最も売れた商品は?}
Excelのピボットテーブル機能を使って,集計期間中に
\begin{itemize}
\item 最も購買回数が多かった商品
\item 最も売上金額の合計が大きかった商品
\end{itemize}
をそれぞれ求めなさい.



\section{ケース2: サイトの認知度向上につき,得られるデータも膨大に!?}
現時点では手持ちのデータは少ないものの,販売履歴データの分析に将来性を感じた山畑さん.
販売履歴データを有効活用できるよう,ショッピングサイト運営により力を入れる決意を固めたのであった.

以下は,山畑さんのその後の話(架空の話)である.

\begin{framed}
ショッピングサイト立ち上げ以降,順調に利用者数も増えていった.
やはりメディアに取り上げられたのが大きかったのだろう.
あのタイミングでサイトの認知度が一気に高まり,サイトの利用者数や利用頻度も加速度的に増えていった.
それに伴い,サイト運営に関わるスタッフも増員した.

販売履歴の管理は,当初は山畑さんが一人で担当していたが,さすがに一人では対応しきれなくなった.
そこで,ある時点から数名体制で販売履歴の記録を行うことになった.
これまで販売履歴の管理に使ってきたExcelシートをクラウドストレージに置き,記録担当スタッフのPC間で同期を取る仕組みを導入.
同じExcelファイルの上で,スタッフ全員で販売履歴を記録できるようにしたのである.

2年後.サイト事業は軌道に乗った.
十分な量の販売履歴データが蓄積されたと判断した山畑さんは,いよいよ大規模な販売履歴データの分析に取りかかることを決意した.
立ち上げ当初は200〜300行しかなかったExcelシートであったが,シートを開きその行数を数えてみると…
なんとその数90万行以上!
データの量に小躍りした山畑さんは,Excelシートの扱いに詳しいスタッフと共に,意気揚々とデータ分析に取りかかったのであった.
\end{framed}


\subsubsection{Q6. データの再確認}
こちらのURL(\url{https://dbnote.hontolab.org/data/purchase\_large.xlsx})から,上のケース2で山畑さんが分析しようとしているExcelファイル\graybox{purchase\_large.xlsx}をダウンロードしなさい.
またダウンロードしたファイルを用いて下記課題(演習1と同じ)に取り組み,データ分析上の課題(困ったこと)を議論しなさい.
以下,課題1の内容を再掲する.
\begin{itemize}
    \item 「岡田 真綾」という人物が何回買い物をしていたかを数えよ
    \item 「ビタミン補助剤」を購入している人をリストアップせよ
    \item 総売上金額を計算せよ
    \item 集計期間中に「最も購買回数が多かった商品」「最も売上金額の合計が大きかった商品」を求めよ
    \end{itemize}
もし,\graybox{purchase\_large.xlsx}ファイルがうまく開けない場合は,こちらのURL(\url{https://dbnote.hontolab.org/data/purchase\_medium.xlsx})からダウンロードできる\graybox{purchase\_medium.xlsx}を用いなさい.
なお,ダウンロードできるExcelシートの構造はケース1で用いた\graybox{purchase\_small.xlsx}と同じである.


\section{おわりに}
ケース1および2で用いたExcelファイルは,販売履歴データの集まりであった.
一般的な認識からすると,このようなデータの集まりは「データベース」ということになるだろう.

ところで,上記演習,とりわけケース2に取り組んでみてイライラしなかっただろうか?
数万件,数十万件ある表データをExcelで扱おうとすると,さまざまな不都合が生じる(図\ref{fig:excel-disaster}).
これは,本来Excelは個人用の表計算アプリケーションであって,大規模データの管理や処理を前提として設計されていないためである.

\begin{figure}[tb]
    \centering
    \includegraphics[width=0.8\textwidth]{figure/excel-disaster.jpg}
    \caption{大きな表データを複数人でExcelで扱うときの悲劇.}
    \label{fig:excel-disaster}
\end{figure}

では,大規模なデータを管理・処理するためにはどうすればよいだろうか?
そのための技術こそが「データベース」である.
以降,本書では\strong{大規模データを効率よく管理・処理するための「データベース」技術} について学習する.


\chapter{データベースの概念}
今日,データベースは様々な業務やアプリケーションで利用されている.
例えば,Amazon\footnote{Amazon.\url{https://www.amazon.co.jp/}}や楽天市場\footnote{楽天市場.\url{https://www.rakuten.co.jp/}}といったオンラインショッピングサイトにおいては,購買履歴管理や在庫管理を行うためにデータベースが用いられている.
Instagram\footnote{Instagram.\url{https://www.instagram.com/}}やTiktok\footnote{Tiktok.\url{https://www.tiktok.com/}}といった,多数のユーザがコンテンツを投稿するソーシャルメディアにおいてもデータベースは欠かせない.
個人用ブログのような比較的小規模なウェブサイトでも,記事データを管理するためにデータベースが用いられている(例: WordPress\footnote{WordPress.\url{https://ja.wordpress.org/}}.
データを語る上で,データベースはなくてはならないものである.

本章では,データベースの概念を説明した後,最も代表的なデータベースであり,かつ本書の主要テーマでもある\strong{関係データベース}の概要について述べる.


% ---------------------------------------
\section{データベースとは}
% ---------------------------------------
データを収集・利用する文化が根付いてきたこともあり,データベースという言葉が市民権を得つつある.
しかし,「データベース」という語が指す意味は,一般人とIT屋とは大きく異なる.
一般人にとってのデータベースとは,「データの集まり」くらいの意味である.
一方,IT屋にとってのデータベースは,
\begin{quote}
``複数の応用目的での共有を意図して,組織的にかつ永続的に格納されたデータ群(北川博之著「データベースシステム」\cite{北川本}より)''
\end{quote}
あるいは
\begin{quote}
``データの正しさを管理する主体によって体系的に整理され,計算機によって永続的に格納されたデータの集まり(吉川正俊著「データベースの基礎」\cite{吉川本}より)''
\end{quote}
を意味する.
特に「データの正しさを組織的に管理する」という概念が重要である.

データベースを扱うためのシステムは,\strong{データベース管理システム(database management system, DBMS)}と呼ばれる.
本来データベースとはDBMSによって管理されるデータ群を指すが,DBMSそのもの(あるいはDBMSとそれによって管理されるデータ群)をデータベース(DB)と呼ぶこともある.

増え続ける多様かつ大量のデータと付き合っていくためには,データの\.う\.ま\.い管理・処理方法が必要になる.
場当たり的にデータを作ってはExcelシートやフォルダ(ディレクトリ)に突っ込んでおくというやり方は,扱うデータが増え,データを使う人間やアプリケーションが増えるにつれて,やがて破綻する.
IT屋が考えるデータベースは,こういった事態を未然に防ぐための強力なツールである.


% ---------------------------------------
\section{データ処理の際に求められる機能}
% ---------------------------------------
大量のデータを管理・処理する際には,以下のような要件が求められる.

\subsubsection{多様かつ大規模なデータの管理}
ビジネスをはじめとする現場では,多種多様かつ大量のデータが時々刻々と発生している.
例えば,2020年1月1日から2020年12月31日までの12か月間に,Amazon.co.jpで購買された商品の数は5億点以上にのぼるとされている\cite{DataVolumeOfAmazon}.

小規模な表データであれば,Excelのような表計算ソフトでも対応できる.
しかし,データの規模が大きくなり,さらに表データの登録・更新を担当する人員が増えていくと…
データの管理が破綻するのは想像に難くない(そもそもExcelは1つの表につき最大100万行程度しか扱えない).
Excelのようなスプレッドシートでは,購買データのように多様かつ大規模なデータを効率よく集積,管理することは困難極まりない(図\ref{fig:heavy-excel}).

\begin{figure}[tb]
    \centering
    \includegraphics[width=1.0\textwidth]{figure/heavy-excel.jpg}
    \caption{管理する表データの規模や数が増えると破綻する.}
    \label{fig:heavy-excel}
\end{figure}


\subsubsection{データの正しさの保証}
管理するデータが大量かつ多様になってくると,データを正しく保つことが難しくなってくる.
管理対象となるデータに誤りが混入すると,大変なことになる.
例えば,図\ref{fig:bad-data}のように,購買データの中に
\begin{itemize}
\item 現実にはあり得ない売り上げ金額が記録されている
\item 入力形式が異なる日付が混在している
\item 取扱商品リストにないはずの商品が購買履歴に含まれている
\end{itemize}
といったことが起きると,オンラインショッピング事業においては一大事である.
それゆえ,データ管理においては,格納されるデータの正しさ,データ間の矛盾のなさを保証する機能が求められる.

\begin{figure}[tb]
    \centering
    \includegraphics[width=1.0\textwidth]{figure/bad-data.jpg}
    \caption{バッドデータ(not ビッグデータ)の例.}
    \label{fig:bad-data}
\end{figure}


\subsubsection{高速で効率的なデータ処理}
データが大量に格納できたとしても,対象となるデータを高速に処理できなければ使い物にならない.
たとえ数百万件の書籍情報を格納しているシステムがあっても,ニーズを満たす書籍リストの検索結果を出力するのに数分待たされるようでは,そのようなシステムは使われない.

ユーザが増えるに従って,システムにかかる負荷も増える.
例えば,Amazon.co.jpでは毎分平均900件以上の商品取引が行われている\cite{TransactionInAmazon}.
このように大量のデータのやりとりが発生するケースにおいても,高速にデータ処理されることが求められる(図\ref{fig:linear-search-for-table}).

\begin{figure}[tb]
    \centering
    \includegraphics[width=1.0\textwidth]{figure/linear-search-for-table.jpg}
    \caption{線形探索では大規模データに対する高速なアクセスは難しい.}
    \label{fig:linear-search-for-table}
\end{figure}


\subsubsection{同時実行(並列処理)}
関連して,大勢の人が同時にデータを検索,追加,更新,削除しても,システムが問題なく動作することも重要である.

例えば,Xさん,Aさん,Bさんがとあるネット銀行の口座を利用しているとしよう.
Xさんの口座残高は100万円であるとする.
振り込みがあった際,振込先口座の残高の計算は
\begin{quotation}
残高 = 振り込み時の「振込先口座」の残高 + 振込額
\end{quotation}
振込元口座の残高の計算は
\begin{quotation}
残高 = 振り込み時の「振込元口座」の残高 - 振込額
\end{quotation}
となる.
これはは至極当然の処理であるが,次のような状況を考えてみよう.

ある日,AさんとBさんがXさんの口座に一秒の狂いもなく,\strong{まったく同時}に10万円送金したとする.
この際,前述の計算式をそのまま適用すると,図\ref{fig:mutual-exclusion}のように誤った処理が行われてしまう.
\begin{figure}[tb]
    \centering
    \includegraphics[width=0.9\textwidth]{figure/mutual-exclusion.jpg}
    \caption{複数の処理要求を矛盾なく処理できていない例.}
    \label{fig:mutual-exclusion}
\end{figure}
AさんとBさんが振込みを行った後,Xさんの口座の残高は110万円になりました -- こんなことが起きたら大変である.
この例における問題点は,たとえ振込み処理が同時に発生したとしても,Aさんの振り込み処理を待ってからBさんの処理を行うべきであったという点である.

このように,データ処理の内容によっては,別の処理が完了したことを保証してから次の処理を行う必要がある.
さらに,処理の途中で何らかのエラーが起きた場合は,すべての処理をキャンセルして最初の状態に戻すことが求められる.
同時にアクセスがあった場合でも,データを矛盾なく処理できることが重要である.


\subsubsection{アクセス権限のコントロール}
データによっては,誰でも自由に閲覧してよいものもあれば,特定の立場のユーザしかアクセスできないようにすべきものも存在する.
多種多様なデータのやりとりが発生する環境においては,ユーザの属性ごとにデータの閲覧,作成,更新,削除といったアクセス権限を制御する機能が必要となる.


% ---------------------------------------
\section{データベースを用いるメリット}
% ---------------------------------------
一人で扱えるほどデータの規模が小さければ,Excelなどの表計算用のソフトウェアでデータ管理しても問題はない.
しかし,扱うデータが多様かつ大規模になると,要求されるデータ処理の質が変わる.
そのため,データの管理方法や処理方法を見直す必要がある.

本稿で学ぶデータベースを用いれば,\strong{データの一元管理}が可能となり,データの管理や利活用において様々な恩恵を受けられる.
具体的には,データベースが備える以下のような特性あるいは関連技術を用いることで,前節で述べた「データ処理に求められる要件」をクリアすることができる.
\begin{itemize}
\item 大規模なデータの管理:「\strong{物理的データ格納方式}」の工夫によって対応
\item データの正しさの保証:「\strong{一貫性制約}」によって対応
\item 高速で効率的なデータ処理:「\strong{索引づけ}」「\strong{問い合わせ最適化}」などで対応
\item 同時実行,データの共同利用:「\strong{トランザクション}」などで対応
\item アクセス権のコントロール:「\strong{ロール管理}」によって対応
\end{itemize}


% ---------------------------------------
\section{データのモデリング}
% ---------------------------------------
\subsubsection{モデリングとは?}
科学やビジネスの世界では,モデルあるいはモデリングという用語がしばしば登場する.
ビジネスプロセスモデル,物理モデル,統計モデルなど,世の中には様々なモデルが存在する.
モデルとは,複雑な仕組みや現象,状態などを表現・分析・操作しやすくするために,本質的でない要素を取り除き,関心のある側面のみを抽出し抽象化したものである.
\strong{モデリング}とはモデル化,つまりモデルを作る行為である.


\subsubsection{データモデリング}
データベースで何らかのデータを扱う場合,まずデータをモデリングする.
世の中に存在するデータは多種多様であり,データを統一的に整理し,計算機で処理しやすい形に抽象化,すなわちモデリングする必要がある.

\strong{データモデリング}とは,データベース化すべき情報を取捨選択し,対象とするデータとその操作に関する枠組み(\strong{データモデル; data model})を設計する行為である.
対象とする事象やアプリケーションに応じて適切なデータモデルを設計することで,データモデルに従って実データを格納し,操作することが可能となる.

一般に,データモデルは以下の3つの要素を含む:
\begin{itemize}
\item データの構造
\item データの制約条件
\item データの操作
\end{itemize}

本講義の主要テーマである\strong{関係データベース(relational database)}は,\strong{関係データモデル(relational data model)}\footnote{関係モデル(relational model)と呼ぶこともある.}に基づき設計されたデータベースである.
関係データモデルの詳細については,次章で説明する.

データベース管理システムで扱われるデータモデルとしては,関係データモデルのほかにも以下のようなものがある:
\begin{itemize}
\item ネットワークモデル\cite{Bachman1969}
\item 階層型データモデル\cite{Tsichritzis1976}
\item オブジェクト指向モデル\cite{Atkinson1994}
\item キー・バリュー(key-value)モデル\cite{DeCandia2007}
\item グラフデータモデル\cite{neo4j}
\end{itemize}

% ---------------------------------------
\section{関係データモデル(導入)}
% ---------------------------------------
関係データモデルは,最も代表的なデータモデルである.
1970年にEdgar F. Codd氏により提案されたもので,単純ながらその背後には強力な数学的基盤をもつ\cite{Codd1970}.
関係データモデルでは,あらゆるデータを\strong{表(table)}としてモデル化する.

図\ref{fig:example-of-relation}は,関係データモデルを用いて構築された,授業の履修状況・成績を管理する関係データベースの例である.
この関係データベースには,以下の3つの表\footnote{テーブルと呼ぶこともある.}が存在する.
\begin{itemize}
\item 学生テーブル:学籍番号,氏名,入学年度,所属といった学生に関する情報を格納
\item 科目テーブル:科目ID,科目名,開講年度といった科目に関する情報を格納
\item 履修テーブル:どの学生が何の科目を履修し,どのような成績であったかに関する情報を格納
\end{itemize}

\begin{figure}[tb]
    \centering
    \includegraphics[width=1.0\textwidth]{figure/example-of-relation.jpg}
    \caption{関係データベースの例.}
    \label{fig:example-of-relation}
\end{figure}
この例だけ見ると「なんだ,関係データモデルとは単なる表なのか」と思われたかもしれないが,\strong{ただの表ではない}.
関係データモデルに基づいて表現された(表)データは,あらかじめ定義された「データの構造」「データの制約」「データの操作」に関する規則に従ってデータが作られ,関係データベース内に格納される.
例えば,以下のような規則が考えられる:
\begin{itemize}
\item 学生テーブルは「学生ID」「姓」「名」「入学年度」「所属」という見出しをもつ
\item 履修テーブルには,科目名や学生の氏名を格納しない
\item 履修テーブルの成績には「優」「良」「可」「不可」のいずれかしか登録できない
\item 履修テーブルに現れる科目IDおよび学生IDは,必ず学生テーブルと科目テーブルに存在する
\item 学生テーブル,科目テーブルの各テーブルにおいて,同じ科目IDは存在しない
\end{itemize}
このような規則に従い,授業の履修状況や成績を管理するための情報が,複数のテーブルに分割され格納される.

次章以降では,
\begin{itemize}
\item このような規則,すなわち関係データモデルをどう設計するのか
\item 一見ただの表にしか見えない関係データモデルが,どのように数学的に定式化されているか
\item 関係データモデルを用いると,なぜ大規模データの管理が効率的になるのか
\end{itemize}
などについて詳しく述べていく.

% ---------------------------------------
\section{クイズ}
% ---------------------------------------
\subsubsection{Q1. メジャーなデータベース管理システム}
ウェブ検索エンジンを用いて,世の中にあるメジャーなデータベース管理システムを調べなさい.
また,調べたデータベース管理システムを「関係データベースを扱うもの」と「そうでないもの」に分類せよ.

\subsubsection{Q2. 線形探索}
100万件ある商品リストの中に特定の商品が含まれているかを確認したい.
商品リストの先頭から末尾まで順に商品名を確認していくと,平均で何回(何件)の確認で商品の有無を確認できるか?

\subsubsection{Q3. データベース管理システムの利用例}
普段利用しているサービスのうち,データベース管理システムを用いていると思われるものを3つピックアップしなさい.

\subsubsection{Q4. CSV/TSVファイル}
CSVファイルおよびTSVファイルとは何かを調べなさい.
また,(データベース管理システムでデータを管理する場合と比較して)CSV/TSVファイルに存在する欠点を挙げなさい.

\subsubsection{Q5. チューリング賞}
関係データモデルを提唱したEdgar F. Codd氏は,計算機科学分野のノーベル賞といわれるチューリング賞の受賞者である.
Codd氏以外で,データベースに関する功績でチューリング賞を受賞した人をピックアップしなさい.

\chapter{関係データモデル}

\strong{関係データベース(relational database)} とは,関係データモデルにもとづき表現されたデータの集まりである.
前章で述べたように,関係データモデルとは,表によってデータを表現するデータモデルである.
一見すると単純な表にしか見えない関係データモデルは数学によって定式化されており,強固な理論的基盤を有している.
関係データモデルを用いることで,データから冗長性を排除し,データを正しく管理することが可能となる.

以下では,数学的な観点から関係データモデルについて述べる.
少々堅い説明にはなるが,関係データベースという計算機科学技術が数学に支えられていることを知るためにも,眠気を我慢して読んでほしい.
なお,本講では数学における「集合と写像」の知識を使う.
%最低限の知識を[コチラ](../misc/math.md)にまとめたので,不安がある方はそちらを参照されたい.


\begin{notebox}{計算機科学とは抽象化の学問}
計算機科学(あるいは情報科学)というと「プログラミングでしょ」と考える学生は多いが,それは間違いである.
計算機科学の本質は,\strong{物事を計算可能な状態にする(抽象化)し,計算機の上で処理・分析}することにある.
データベースもその一つである.
計算機科学は抽象化の学問なのである.
\end{notebox}

% ---------------------------------------
\section{関係データモデルのデータ構造}
% ---------------------------------------
\begin{table}[tb]
\centering
\caption{ある小売店の商品に関するデータを収めた関係}
\label{tab:correct-table}
\begin{tabular}{@{}llll@{}}
商品            &             &             &              \\ \midrule
\textbf{商品ID} & \textbf{名称} & \textbf{単価} & \textbf{登録日} \\ \midrule
P1            & はーいお茶       & 130         & 2020/07/15   \\
P2            & 午前の紅茶       & 130         & 2020/09/25   \\
P3            & 健康麦茶        & 150         & 2021/02/16   \\
$\vdots$      & $\vdots$    & $\vdots$    & $\vdots$     \\
P1000         & きのこの里       & 200         & 2023/01/08   \\ \bottomrule
\end{tabular}
\end{table}

表\ref{tab:correct-table}は,ある小売店で取り扱っている商品のデータを関係データモデルによって表現したものである.
この表において,各行は小売店で取り扱っている商品に対応し,各列には各商品に関するデータが記述されている.

表の左上にある「商品」はこの表の名前を示しており,\strong{関係名}と呼ばれる.
関係データモデル上では,各行は\strong{タプル(tuple)} と呼ばれる.
表の1行目には,「商品」がどのようなデータを持つかを示す見出しが付けられている.
表における各見出しに対応するものを,関係データモデルでは\strong{属性(attribute)}と呼ぶ.
また,各タプルにおけるある属性の値を\strong{属性値}と呼ぶ.
例えば,表中の2行目に対応するタプルは
\begin{equation}
(``P1", ``はーいお茶", 130, ``2020/07/15")
\end{equation}
であるが,このタプルの属性「単価」の属性値は\graybox{130}となる.

% ---------------------------------------
\subsection{数学における「関係」}
% ---------------------------------------
ところで,関係データベースあるいは関係データモデルの「関係」とは一体何だろうか.
\strong{関係(relation)} とは数学上の概念である.
集合$S_1, S_2, ..., S_n$が与えられたとき,それらの直積集合$S_1 \times ... \times S_n$とは

\begin{equation}
S_1 \times ... S_n = \{(x_1, ..., x_n) \ | \ x_1 \in S_1, ..., x_n \in S_n \}
\end{equation}
と表現されるものである.
(数学上の概念である)関係とは,このような直積集合の部分集合を意味する.
正確に書くと,集合$S_1, S_2, ..., S_n$が与えられたとき,直積集合$S_1 \times ... \times S_n$の部分集合を$S_1, S_2, ..., S_n$上の\strong{n項関係}と呼ぶ.

例えば,$A=\{2, 3\}$,$B=\{-2, -3\}$が与えられたとき,直積$A \times B$は集合AおよびBの要素のすべての組み合わせであるから

\begin{equation}
A \times B = \{(2, -2), (2, -3), (3, -2), (3, -3)\}
\end{equation}
となる.
このとき,その部分集合の1つである
\begin{equation}
\{(2, -3), (3, -2)\}
\end{equation}
は集合AおよびB上の2項関係となる.
このような(n項)関係の概念を使って,関係データモデルは理論構築されている


% ---------------------------------------
\subsection{関係データモデルにおける「関係」}
% ---------------------------------------
\strong{関係データモデルにおける関係}は「数学におけるの関係」を拡張したものになっており,以下の2つから構成される.
\begin{itemize}
\item 関係スキーマ
\item インスタンス
\end{itemize}

\strong{関係スキーマ(relation schema)} とは,関係の名前と属性の集合,および(後述する)一貫性制約の情報を示すものである.
関係スキーマは

\begin{equation}
(\boldsymbol{R}(A_1, ..., A_n), \{\sigma_1, \sigma_2, ..., \sigma_m\})
\end{equation}
の形式で記述される.
ここで
\begin{itemize}
\item $\boldsymbol{R}$は関係名
\item $A_1, ..., A_n$は属性
\item $\sigma_1, ..., \sigma_m$は一貫性制約
\end{itemize}
に対応する.
一貫性制約が自明な場合やそれを考慮しないときは,関係スキーマを

\begin{equation}
\boldsymbol{R}(A_1, ..., A_n)
\end{equation}
のように簡潔に表記することもある.
表\ref{tab:correct-table}の関係「商品」の例の場合,関係スキーマは
\begin{equation}
商品(商品ID, 名称, 単価, 登録日)
\end{equation}
と記述する.

関係スキーマに記された各属性には,属性が取り得る値の集合を定義する.
この集合は\strong{ドメイン(domain; 定義域)} と呼ばれ,属性$A$のドメインは$Dom(A)$と記す.
例えば,表\ref{tab:correct-table}の関係「商品」の例では,属性「単価」は金額を表すので,そのドメイン$Dom(単価)$は0以上の整数値が想定される.
これを数学的に記すと,以下のようになる:
\begin{equation}
Dom(単価) = \{ x \ | \ x \in \mathbb{N} \}
\end{equation}

同様に,属性「商品ID」「名称」「登録日」についても,以下のようにドメインを定義できる.
\begin{eqnarray}
Dom(商品ID) = \{ x \ | x \in Pから始まる文字列集合 \} \\
Dom(名称) = \{ x \ | \ x \in 文字列集合 \} \\
Dom(登録日) = \{ x \ | \ x \in 年月日集合(ただし西暦表記) \}
\end{eqnarray}

\strong{インスタンス(instance)} は,関係スキーマ$\boldsymbol{R}(A_1, ..., A_n)$におけるドメイン$Dom(A_1), ..., Dom(A_n)$の直積の部分集合である.
仮にインスタンスを$R$とすると,
\begin{equation}
R \subset{Dom(A_1) \times ... \times Dom(A_n)}
\end{equation}
である.
インスタンスは,前節で説明した数学上の関係に相当する.
また,インスタンス$R$の要素がタプルとなる.

例に戻って,定義と照らし合わせてみよう.
表\ref{tab:correct-table}の関係「商品」の関係スキーマは$商品(商品ID, 名称, 単価, 登録日)$であった.
各属性のドメインの定義も与えたが,その定義から,例えば属性「単価」については
\begin{equation}
Dom(単価) = \{1, 2, 3, .... 10000, .... \}
\end{equation}
のようにあらゆる自然数を取り得る.
また,属性「名称」については
\begin{equation}
Dom(名称) = \{``あ", ``い", ..., ``はーいお茶", ... \}
\end{equation}
のようにあらゆる文字列を取り得る.

各属性の直積$Dom(商品ID) \times Dom(名称) \times Dom(単価) \times Dom(登録日)$には,
\begin{equation}
(``P001", ``こんな商品はありません", 100兆, ``2020/01/01")
\end{equation}
のように,実際に商品情報としてありえないタプルも含めて,様々なタプルが要素として含まれる.
なぜなら,直積とは集合(ドメイン)の要素のすべての組み合わせであるためである.
ドメインの直積の部分集合をインスタンスとすることは,取り得るタプルの候補から実際にデータとして扱われるタプルの集合を選択することに相当する.


\begin{notebox}{タプルは重複が許されない}
一般的に使われる表では,同じ内容を表す行が複数あっても許される.
一方,関係データモデルに基づく表は行,すなわちタプルの重複は許されない.
なぜなら,関係データモデルのインスタンスは(数学的な)集合だからである.
\end{notebox}


% ---------------------------------------
\section{非正規関係と第1正規形}
% ---------------------------------------
関係データモデルにおけるドメインとは属性が取り得る値の集合を意味するが,ドメインの要素はそれ以上分解不可能な値を意味する\strong{原子値(atomic value)} であることを想定している.
原子値の例としては,文字列,整数,実数,真偽値,日付などが挙げられ,どれもデータの基本単位と呼べるものである.
逆に原子値で「ない」例としては,タプルや集合,関係などが挙げられる.
ドメインとして原子値以外の値をとることを許す関係データを\strong{非正規関係(unnormalized relation)} と呼ぶ.
逆に,ドメインとして原子値しかとらない関係は\strong{第1正規形(first normal form; 1NF)である} という.

\begin{figure}[tb]
    \centering
    \includegraphics[width=1.0\textwidth]{figure/unnormalized-relation.jpg}
    \caption{非正規関係}
    \label{fig:unnormalized-relation}
\end{figure}

例えば,図\ref{fig:unnormalized-relation}の(a)(b)はともに商品の購買履歴を表しているが,(a)は属性「購入品目」に集合要素が入ってしまっている(例えば,\graybox{\{``はーいお茶'', ``きのこの里'', ``のど飴''\}}).
(a)の属性「購入品目」のドメインが原子値以外の値(つまり文字列の集合)を許してしまっていることから,(a)は非正規関係と見なせる.

一方,図\ref{fig:unnormalized-relation}(b)は(a)の購買品目の値が原子値(文字列の集合ではなく文字列)になるように設計されているため,第1正規形と見なせる.
図\ref{fig:unnormalized-relation}(a)では1行で表現できていた1回分の購買データが,(b)では複数行を使って表現されている.
図\ref{fig:unnormalized-relation}(b)のほうがデータ構造が簡潔になるため,データが扱いやすい.


% ---------------------------------------
\section{一貫性制約}
% ---------------------------------------
\strong{一貫性制約(integrity constraint)} とは,データベースが対象とする実世界を反映するように設定された,データが満たすべき規則である.
関係データモデルにおいては,代表的な一貫性制約として以下がある:
\begin{itemize}
\item ドメイン制約
\item キー制約
\item 参照制約
\item データ従属性
\end{itemize}

関係データモデルを扱うデータベースを設計する際には,上記一貫性制約を踏まえて\strong{関係スキーマの定義や(複数の関係スキーマへの)分解} を行う.


% ---------------------------------------
\subsection{ドメイン制約}
% ---------------------------------------
\strong{ドメイン制約(domain constraint)} とは,関係$\boldsymbol{R}(A_1, ..., A_n)$に含まれるタプルの各成分は,対応する属性のドメインの要素でなければならならいという制約である.
これは属性の定義ですでに触れたことである.
ドメイン制約では,各属性のドメインのデータ型(例: 整数,実数,文字列,日付)に加えて,値が取り得る範囲を指定することもある.
例えば,

先の例でとりあげた関係
\begin{equation}
商品(商品ID, 名称, 単価, 登録日)
\end{equation}
において,属性「単価」や「登録日」のドメインは
\begin{eqnarray}
Dom(単価) = \{ x \ | \ x \in \mathbb{N} \} \\
Dom(登録日) = \{ x \ | \ x \in 年月日集合(ただし西暦表記) \}
\end{eqnarray}
のように定義されていた.
これらはドメイン制約である.
これら制約にもとづき,
\begin{itemize}
\item 「単価」属性の値は\strong{必ず}自然数
\item 「登録日」属性の値は\strong{必ず}西暦表記の年月日集合
\end{itemize}
でなければならない.
よって,表\ref{tab:incorrect-table}に記した表「商品(*)」の各タプルは関係「商品(商品ID, 名称, 単価, 登録日)」のインスタンスには\strong{なれない}.
なぜなら,
\begin{itemize}
\item 商品IDがP1の商品の登録日は年月はあるが日が抜けている
\item 商品IDがP3の商品の登録日は和暦で表現されている
\item 商品IDがP1000の商品の単価は漢数字(文字列)で表現されている
\end{itemize}
からである.

\begin{table}[tb]
\centering
\caption{ドメイン制約に違反する関係}
\label{tab:incorrect-table}
\begin{tabular}{@{}llll@{}}
商品(*)            &             &             &              \\ \midrule
\textbf{商品ID} & \textbf{名称} & \textbf{単価} & \textbf{登録日} \\ \midrule
P1            & はーいお茶       & 130         & 2020/07   \\
P2            & 午前の紅茶       & 130         & 2020/09/25   \\
P3            & 健康麦茶        & 150         & 令和3年2月16日   \\
$\vdots$      & $\vdots$    & $\vdots$    & $\vdots$     \\
P1000         & きのこの里       & 2百         & 2023/01/08   \\ \bottomrule
\end{tabular}
\end{table}


% ---------------------------------------
\subsection{キー制約}
% ---------------------------------------
まず,キー制約の前提となるキーの概念について説明する.
関係$\boldsymbol{R}$における\strong{超キー(super key)} とは,関係$\boldsymbol{R}$における属性の集合のうち,それらの属性値が決まればおのずと関係$\boldsymbol{R}$のタプルが唯一ひとつに決まる(タプルを一意に特定できる)ものを指す.

例えば,表\ref{tab:contact-relation}は関係スキーマ
\begin{equation}
連絡先(学生ID, 名前, 学年, 大学email, 自宅住所)
\end{equation}
に従うデータを表形式で記したもので,各行(タプル)は学生の連絡先情報を示している.
大学において学生情報を管理する場合,学生ID(の値)が決まれば唯一ひとつの学生連絡先情報(行; タプル)を照会できるよう,学生IDは重複がないように設定される.
それゆえ,この関係「連絡先」において属性「学生ID」は超キーとなる.
同様に,属性「大学email」も超キーとなる.

なお,\{ 学生ID, 名前 \},\{ 学生ID, 名前, 学年 \} のように学生IDを含む属性の集合も超キーになる.
例えば,「学生ID」が\graybox{S1},「名前」が\graybox{川澄桜}である行は唯一1行に決まるので,属性集合 \{ 学生ID, 名前 \} も超キーになる.
これは属性「学生ID」が超キーであるから,当たり前である.

一方,属性「住所」は超キーにはなり得ない.
学生IDが\graybox{S3}と\graybox{S4}の学生の住所が同じ(つまり同じところに住んでいる)であるから,住所を1つ指定しても連絡先タプルが一意に特定できないからである.


\begin{table}[tb]
\centering
\caption{関係「連絡先」}
\label{tab:contact-relation}
\begin{tabular}{@{}lllll@{}}
連絡先           &             &             &                    &               \\ \midrule
\textbf{学生ID} & \textbf{名前} & \textbf{学年} & \textbf{大学email}   & \textbf{自宅住所} \\ \midrule
S1            & 川澄 桜        & 4           & kawasumi@xxx.ac.jp & 名古屋市xxx       \\
S2            & 山畑 滝子       & 3           & taki@xxx.ac.jp     & 岡崎市xxx        \\
S3            & 田辺 通        & 3           & t.tanabe@xxx.ac.jp & 尾張旭市xxx       \\
S4            & 田辺 瑞穂       & 1           & m.tanabe@xxx.ac.jp & 尾張旭市xxx       \\
$\vdots$      & $\vdots$    & $\vdots$    & $\vdots$           & $\vdots$      \\
S1000         & 北 千種        & 2           & kita@xxx.ac.jp     & 名古屋市xxx       \\ \bottomrule
\end{tabular}
\end{table}


超キーは複数ありえるが,上の例では属性「学生ID」のみでタプルを特定できるように,属性集合 { 学生ID, 名前 } を使ってタプルを特定しようとするのは無駄であろう.
そこで,超キーのうち極小(つまり最も小さい部分集合)のものを\strong{候補キー(candiate key)} と定義する.
候補キーは単純に\strong{キー(key)} と呼ばれることもある.
例えば,上の関係「連絡先」では,属性「学生ID」や「大学email」が候補キーとなりえる.

最後に\strong{主キー(primary key)} の概念を導入する.
候補キーのうち,値として\strong{未定義や空の値(空値またはNULL値と呼ばれる)} を取る可能性がなく,かつデータベース管理上都合のよいキーの1つを主キーと決める.
主キー以外の候補キーは\strong{代替キー}と呼ばれる.
例えば,先の関係「連絡先」の場合,候補キーは「学生ID」と「大学email」であったが,大学emailは「氏名\@xxx.ac.jp」のようなものから「学籍番号\@xxx.ac.jp」のようなものに変更される可能性があったとしても,学生IDは一度決めたらほぼ変わらないと思われる.
これらを鑑みると,関係「連絡先」においては属性「学生ID」を主キーとして選ぶのがよさそうである.

なお,候補キーの定義から主キーは属性の集合をとることができる.
このような主キーを\strong{複合主キー(composite primary key)} と呼ぶ.
例えば,表\ref{tab:course-completion-relation}は,学生が履修した科目の成績をあらわす関係「履修」を表にしたものである.
この表においては科目IDあるいは学生IDだけでは成績が特定できないが,科目IDと学生IDの両方が決まれば,ある学生のある科目の成績が特定される.
それゆえ,関係「履修」においては \{ 科目ID, 学生ID \} の属性ペアが主キーとなる.

\begin{table}[tb]
\centering
\caption{関係「履修」}
\label{tab:course-completion-relation}
\begin{tabular}{@{}lll@{}}
履修            &               &             \\ \midrule
\textbf{科目ID} & \textbf{学生ID} & \textbf{成績} \\ \midrule
C1            & S1            & 不可          \\
C2            & S1            & 良           \\
C2            & S2            & 優           \\
C3            & S3            & 可           \\
$\vdots$      & $\vdots$      & $\vdots$    \\ \bottomrule
\end{tabular}
\end{table}

関係表と主キー,候補キー,超キーの関係は,図\ref{fig:key-concept-mapping}のようにまとめることができる.
\begin{figure}[tb]
    \centering
    \includegraphics[width=1.0\textwidth]{figure/key-concept-mapping.jpg}
    \caption{キーの概念の整理}
    \label{fig:key-concept-mapping}
\end{figure}

ここまで,超キー,候補キー,主キーの説明を行ったが,最後に本節の主題であるキー制約について述べる.
\strong{キー制約(key constraint)} は,関係$\boldsymbol{R}$の関係スキーマに対して主キーが設定されたとき,$\boldsymbol{R}$のインスタンスにおいて
\begin{itemize}
\item 主キーに設定された属性の値は重複があってはならない,かつ
\item その要素(タプル)は主キーによって一意に特定されなければならない,かつ
\item 主キーとなる属性の値は\strong{NULL値} であってはならない
\end{itemize}
という制約である.
定義は小難しく見えるかもしれないが,キー制約は「ある属性が主キーと設定されたら,それがきちんと主キーの役割を果たすようデータが作られなければならない」ということを意味している.
主キーを定義できれば,キー制約を定義できたようなものである.

さて,関係スキーマにおいてキー制約以外の一貫性制約には注目しない場合,関係スキーマでは主キーは以下のように下線を引いて表す.
\begin{equation}
\boldsymbol{R}(\underline{A_1, A_2}, ..., A_n)
\end{equation}
例えば,図\ref{fig:key-concept-mapping}の関係「連絡先」であれば,その関係スキーマは以下のように表す.
\begin{equation}
連絡先(\underline{学生ID}, 名前, 学年, 大学email, 自宅住所)
\end{equation}

\begin{notebox}{キーは関係スキーマに対して設定される}
キーは,関係データベース内にあるタプル(行)だけを見て決めてはいけない.
例えば表\ref{tab:contact-relation}に示した関係「連絡先」の場合,表の見えているところだけに注目すると,属性「名前」もキーに見える.
しかし,現実的には同性同名の学生が存在しうる.
そのため,「名前」をキーにすると,連絡先のタプルを一意に特定できない可能性が生じる(キーの性質を失う).

そもそも関係データモデルにおいては,どのようなデータをデータベースに格納する可能性があるかを事前に想定して関係スキーマを設定する(その中にキーの設定も含まれる).
その上で,設定した関係スキーマに従ってデータを生成し格納する.
決してインスタンスを見てキーを決めるわけではない.
キーは「\strong{関係スキーマに対して設定される}」ことを意識しよう.
\end{notebox}


% ---------------------------------------
\subsection{参照整合性制約}
% ---------------------------------------
(関係データモデルにもとづく)関係データベースでは,対象となる事象のデータを正しく管理するために,\strong{データを複数の関係(表)で管理する}ことがほとんどである.
複合主キーの説明の例で使った関係「履修」(表\ref{tab:course-completion-relation}参照)には,その値だけ見ても具体的に何を意味しているのかが分からない属性「科目ID」「学生ID」がある.
これらの意味を読み解くためには
\begin{itemize}
\item 「科目ID」がどのような科目のことを指しているのか
\item 「学生ID」がどの学生のことを指しているのか
\end{itemize}
といった情報を管理する他の関係(表)が必要となる.
図\ref{fig:score-db}は,学生の成績を管理するための関係データベースの例である.
この中には関係「履修」も含まれている.
\begin{figure}[tb]
    \centering
    \includegraphics[width=1.0\textwidth]{figure/score-db.jpg}
    \caption{成績管理データベース}
    \label{fig:score-db}
\end{figure}
図中の3つの関係(表)があれば,
\begin{itemize}
\item 関係「学生」から「学生ID」が\graybox{S1}のものを探し,
\item 関係「科目」から「科目ID」が\graybox{C1}のものを探す
\end{itemize}
ことで,関係「履修」において「学生ID」が\graybox{S1}で「科目ID」が\graybox{C1}の「成績」が\graybox{不可}だった件について,具体的にどんな学生が何の科目を落としてしまった(不可になってしまった)かを把握することができる.
このとき,関係「履修」のタプルにある「科目ID」の\graybox{C1}という値が,関係「科目」のインスタンスの属性「科目ID」に含まれていなければならない(関係「科目」の科目IDの列にその値が存在しなければならない),ということは自明であろう.
関係「履修」の中にはあるが,関係「科目」には存在していなければ,データ管理として破綻していることになる.
「学生ID」の\graybox{S1}についても同様である.

上記の例では,
\begin{itemize}
\item 関係「履修」の属性「学生ID」は関係「学生」の主キーである「学生ID」と
\item 関係「履修」の属性「科目ID」は関係「科目」の主キーである「科目ID」と
\end{itemize}

紐付いていることが分かる.
言い方を変えると,関係「履修」は属性「学生ID」および「科目ID」の値を通して,関係「学生」および「科目」の情報を参照していることになる.

このように,関係スキーマ$\boldsymbol{R_1}(..., FK, ...)$と$\boldsymbol{R_2}(\underline{PK}, ...)$が与えられ,$\boldsymbol{R_1}$におけるタプルの属性$FK$の値は必ず$\boldsymbol{R_2}$におけるいずれかのタプルの主キー$PK$の値と一致するように設計されているとき,$\boldsymbol{R_1}$の属性$FK$を$\boldsymbol{R_2}$の$PK$に対する\strong{外部キー(foreign key)} と呼ぶ.
関係スキーマ上では,関係$\boldsymbol{R_1}$の属性$FK$が関係$\boldsymbol{R_2}$の主キー$PK$に対する外部キーであることを以下のように記す.
\begin{equation}
\boldsymbol{R_1}.FK \subseteq \boldsymbol{R_1}.PK
\end{equation}

図\ref{fig:score-db}の例では,関係「履修」の属性「学生ID」および「科目ID」が外部キーとなるので,関係スキーマとして
\begin{eqnarray}
履修.学生ID \subseteq 学生.学生ID \\
履修.科目ID \subseteq 科目.科目ID
\end{eqnarray}
と記す.

前置きが長くなったが,本節のテーマである\strong{参照整合性制約(referential constraint)} とは「関係スキーマにおいて外部キーが設定されたとき,そのいかなるインスタンスも設定された外部キーの条件を満たさなければいけない」という制約である.

% ---------------------------------------
\subsection{データ従属性}
% ---------------------------------------
一貫性制約として,ドメイン制約,キー制約,参照制約を挙げてきた.
関係データモデルにはこれら以外にも,データ間に成立する制約を数学的に記述する手段がある.
これを\strong{データ従属性(data dependency)} と呼ぶ.

データ従属性としては,関数従属性,多値従属性,結合従属性など,様々なものが提案されている.
\strong{関数従属性(functional dependency)} は,データ従属性の中でも最も単純で,かつ実用的なものである.
関数従属性は,「関係$\boldsymbol{R}(..., X, ..., Y, ...)$において,属性(あるいは属性集合)$X$の値が決まると属性$Y$の値も一意に決まる」という性質である.
このことを
\begin{equation}
X \to Y
\end{equation}
と記す.

例えば,関係として
\begin{equation}
連絡先(\underline{学生ID}, 名前, 学年, 大学email, 郵便番号, 都道府県, 自宅住所)
\end{equation}
が与えられたとしよう.
一般常識から,住所が決まればそれが存在する都道府県や郵便番号もひとつに決まる.
このことから,関係「連絡先」においては
\begin{eqnarray}
自宅住所 \to 郵便番号 \\
自宅住所 \to 都道府県
\end{eqnarray}
という関数従属性が定義できる.
関数従属性はキー制約を一般化したものと捉えることができる.

データ従属性は,データの更新があったときにその影響が最小限になるように関係データベースを設計する上で,極めて重要である.
それゆえ,特に関数従属性については別章で取り上げる.


% ---------------------------------------
\section{クイズ}
% ---------------------------------------
\subsubsection{Q1. 直積}
集合$S_{lang}$,$S_{popularity}$,$S_{difficulty}$を以下のように定義する:
\begin{eqnarray}
S_{lang} = \{Python, R, C^{++}\} \\
S_{popularity} = \{人気, 不人気\} \\
S_{difficulty} = \{難, 普通, 易\}
\end{eqnarray}
このとき,直積集合$S_{lang} \times S_{popluarity} \times S_{difficulty}$の要素をすべて列挙せよ.

\subsubsection{Q2. 関係}
Q1で定義した集合$S_{lang}$,$S_{popularity}$,$S_{difficulty}$上の3項関係を適当に考えよ
(※ 正解は一意に決まらないので,深く悩まずクイズに取り組むこと).


\subsubsection{Q3. 関係スキーマ}
関係スキーマ
\begin{equation}
学生(\underline{学籍番号}, 氏名, 学部, 年齢, 出身都道府県)
\end{equation}
に従う表データの例を作成せよ.
なお,表の行数は見出し行を含めて5-6行程度でよい.


\subsubsection{Q4. ドメイン}
Q3で定義した関係スキーマ「学生」の各属性について,そのドメインを定義せよ.
\begin{equation}
Dom(学籍番号) = \{ ... \}
\end{equation}
の形式で頑張って書いてみること.

\subsubsection{Q5. ドメイン制約}
Q4で定義した関係スキーマ(ドメイン定義を含む)に対して,ドメイン制約に違反しているタプルの例を2,3個列挙せよ.


\subsubsection{Q6. 候補キー(令和4年度 ITパスポート試験 問65改題)}
関係スキーマ
\begin{equation}
従業員(従業員番号, 従業員名, 部門コード, 生年月日, 住所)
\end{equation}
において,候補キーは何か.
なお,関係「従業員」は以下のような制約条件をもつ:
\begin{itemize}
\item 各従業員は重複のない従業員番号を1つだけもつ
\item 同姓同名の従業員がいてもよい
\item 各部門は重複のない部門コードを1つだけもつ
\item 1つの部門には複数名の従業員が所属する
\item 1人の従業員が所属する部門は1つだけである
\end{itemize}


\subsubsection{Q7. 主キー}
表\ref{tab:membership-management}の「会員管理」において,想定される主キーは何か.

\begin{table}[tb]
\centering
\caption{関係「会員管理」}
\begin{tabular}{@{}lllll@{}}
会員管理 &            &             &               &               \\ \midrule
\textbf{店舗コード} & \textbf{店舗名} & \textbf{会員番号} & \textbf{会員名} & \textbf{会員種別} \\ \midrule
S01            & 星ヶ丘          & 1             & 山畑 滝子        & ゴールド          \\
S01            & 星ヶ丘          & 2             & 田辺 通         & ゴールド          \\
S02            & 八事           & 1             & 北 千種         & 学生            \\
S02            & 八事           & 2             & 川澄 桜         & プラチナ          \\
S02            & 御器所          & 1             & 川澄 桜         & 一般            \\ \bottomrule
\end{tabular}
\label{tab:membership-management}
\end{table}


\subsubsection{Q8. 参照整合性制約}
以下の関係スキーマをもつ5つの関係からなるデータベースにおいて,定義すべき参照整合性制約をあげよ.
\begin{eqnarray}
顧客(\underline{顧客ID}, 氏名, 性別) \\
店舗(\underline{店舗ID}, 店舗名, 住所) \\
商品(\underline{商品ID}, 商品名, 商品カテゴリID, 単価) \\
購買(\underline{購買ID}, 店舗ID, 顧客ID, 商品ID, 個数, 購買日) \\
商品カテゴリ(\underline{商品カテゴリID}, カテゴリ名)
\end{eqnarray}

%\chapter{SQL}
データベースからデータを引っ張ってくるなど,データベースを操作するには具体的なツールが必要となる.
\strong{SQL(Structured Query Language)} は関係データベースの操作に特化した言語である.
SQLはISOによって国際的に標準化されている.
よく用いられる関係データベース管理システム(RDBMS)としてMySQL\footnote{\url{https://ja.wikipedia.org/wiki/MySQL}}やPostregreSQL\footnote{\url{https://ja.wikipedia.org/wiki/PostgreSQL}},Oracle Database\footnote{\url{https://ja.wikipedia.org/wiki/Oracle_Database}},Microsoft SQL Server\footnote{\url{https://ja.wikipedia.org/wiki/Microsoft\_SQL\_Server}}などが挙げられるが,どのRDBMSにもSQLは実装されている.

SQLを使うことで,関係データベースのユーザは
\begin{itemize}
\item データの登録(Create)
\item データの読み出し(Read)
\item データの更新(Update)
\item データの削除(Delete)
\end{itemize}

が可能となる\footnote{データの作成,読み出し,更新,削除はデータベースに求められる主要な操作で,それぞれの頭文字を取ってCRUD(クラッド)と呼ばれる.}
なお,本教材のターゲットはデータ分析人材の卵であることから,本教材では関係データベースから\strong{データの読み出し}に焦点を当ててSQLの説明を行う.

SQLはデータベースに対する問い合わせ言語であって,プログラミング言語ではない.
そのため,プログラミング言語に比べて覚えることは少なく,問い合わせ内容を英語に近い形で書き下すことができるよう設計されている.
データ分析人材はソフトウェアを開発することが目的ではないから,高度なプログラミング能力を持つ必要は必ずしもない.
しかし,関係データベースに納められた大量のデータを自由自在に扱うためにも,\strong{SQLの習得は必須}である.


\begin{notebox}{SQLの読み方}
SQLは「エス・キュー・エル」と読む.SQLは1970年代にIBM社が開発したデータベース管理システムSystem Rの操作言語SEQUELを起源としている.関係データベースに熱狂していた世代の人の中には,SEQUELの名残でSQLのことを「シークエル」と呼ぶ人もいる.
\end{notebox}

\begin{warningbox}{SQLにも方言がある}
RDBMSによって,SQLで使用できる関数が異なったりすることがある.特にSQLiteについては,簡易的なRDMBSということもあり,他のRDBMSでは実装されている機能や関数が実装されていない場合がある.自分が書いたSQL文がうまく動作しない場合は,マニュアルに当たってみよう.
\end{warningbox}


\section{SQLと関係データモデルの比較}
SQLは関係データベースを操作する言語であり,それが扱うデータモデルは関係データモデルを実用的に拡張したものである.
「実用」のSQLデータモデルと「理論」の関係データモデルでは,同様の概念が異なる用語で定義されている.
以下,SQLと関係データモデルの用語の比較である.

\begin{table}[tb]
    \centering
    \caption{SQLと関係データモデルの用語の比較}
    \begin{tabular}{ll}
    \toprule
    \textbf{関係データモデル}             & \textbf{SQL}            \\ \midrule
    関係(リレーション; relation) & 表(テーブル; table) \\
    タプル(tuple)           & レコード or 行(row) \\
    属性(attribute)        & 列(カラム; column) \\
    定義域(ドメイン; domain)    & データ型           \\ \bottomrule
    \end{tabular}
    \end{table}

また,関係データモデルでは各属性はドメインで定義された値をもつが,
現実的にはある属性の値が存在しない,あるいは未定義(不明)のケースがありえる.
そのような状況では,特殊な値である\strong{NULL値}(ヌルと読む)を用いる.


\section{基本形}
関係データベースに対する最も典型的な問い合わせ要求は,
\begin{itemize}
\item 特定の\strong{表(table,テーブル)}から
\item \strong{列(column,カラム)} が特定の条件を満たす\strong{行(row)}\footnote{レコード(record)と呼ぶこともある.} を見つけ出し
\item 結果を表形式で出力する
\end{itemize}
ことである.
問い合わせのために記述するSQL文のことを\strong{クエリ(query)} と呼ぶ.

SQLによる代表的なクエリは,コード\ref{code:first-sql}に記したようなの形式の\strong{SELECT文}である.
\begin{figure}[tb]
    \begin{minted}[gobble=1,
        bgcolor=background,
        fontsize=\small,
        linenos=false,
        xleftmargin=0em]{sql}

    SELECT
        都道府県名
    FROM
        都道府県
    WHERE
        人口 >= 5000000;
    \end{minted}
    \captionsetup{name=コード}
    \caption{単純なSELECT文}
    \label{code:first-sql}
\end{figure}
上記クエリは,
\begin{framed}
「都道府県」テーブルから「人口」という列の値が500万以上である行を見つけ,その行の列「都道府県名」を出力してください
\end{framed}
という問い合わせ要求を表現したものである.
これら要求を表現するために,コード\ref{code:first-sql}のクエリでは
\begin{itemize}
\item \textsc{SELECT}句を用いて「問い合わせ結果に表示したい列情報」,
\item \textsc{FROM}句を用いて「参照したい表」,
\item \textsc{WHERE}句を用いて「表示する際の条件」
\end{itemize}
を指定している.
上記要求を英語に直してみると,SQL文は問い合わせ要求をできるだけ直感的に表現しようとしていることがお分かりいただけると思う.

コード\ref{code:first-sql}のクエリは単純な例ではあるが,関係データベースからデータを引っ張ってくる問い合わせについては,すべからくコード\ref{code:sql-structure}のような\textsc{SELECT}句から始まるクエリが用いられる.
\begin{figure}[tb]
    \begin{minted}[gobble=1,
        bgcolor=background,
        fontsize=\small,
        linenos=false,
        xleftmargin=0em]{sql}

    SELECT
        列名1, 列名2, ...
    FROM
        参照する表1, 表2, ...
    [WHERE 条件]
    [GROUP BY 列名1, 列名2, ...]
    [HAVING 条件]
    [ORDER BY 列名1, 列名2, ...]
    [LIMIT 数字];
    \end{minted}
    \captionsetup{name=コード}
    \caption{SQLの構文}
    \label{code:sql-structure}
\end{figure}
このうち,問い合わせ結果に表示したい情報を指定するSELECT句,および問い合わせの際に参照するテーブルを指定しているFROM句は必須である.
[ ]で囲まれた箇所については,書かなくてもSQLとして動作する.
なお,SELECTやFROM,WHERE,GROUP BY,HAVING,ORDER BY,LIMITといった句の意味のついては後ほど説明するが,各句は\strong{この順序で使う}必要がある.
例えば,HAVING句はWHERE句の後ろに書かない.
また,クエリの末尾にはピリオド(;)をつけ忘れてはいけない.

以後,SQLの基礎について説明する.
説明には,独立行政法人統計センターが公開している教育用標準データセット(SSDSE)の基本素材SSDSE-E\footnote{\url{https://www.nstac.go.jp/use/literacy/ssdse/\#SSDSE-E}}(データの解説はこちら)から抜粋・加工したデータ(populationテーブル)を用いる.
表\ref{tab:population-table}の通り,populationテーブルには,47ある各都道府県に関する総人口,小学校児童数,中学校生徒数,高等学校生徒数,大学学生数のデータが2021年度,2020年度分格納されている.
このテーブルが格納された関係データベースが手元にあると想定して,SQLの使い方を説明する.
\begin{table}[tb]
    \centering
    \caption{SSDSE-Eから抜粋・加工したデータ: populationテーブル}
    \scalebox{0.73}{
    \begin{tabular}{@{}rrrrrrrr@{}}
    \toprule
    \textbf{地域コード} & \textbf{都道府県} & \textbf{調査年度} & \textbf{総人口} & \textbf{小学校児童数} & \textbf{中学校生徒数} & \textbf{高等学校生徒数} & \textbf{大学学生数} \\ \midrule
    R01000 & 北海道 & 2021 & 5183000 & 231714 & 122742 & 115335 & 79729 \\
    R02000 & 青森県 & 2021 & 1221000 & 54460  & 29940  & 30543  & 15419 \\
    R03000 & 岩手県 & 2021 & 1196000 & 55597  & 30269  & 29980  & 11340 \\
    R04000 & 宮城県 & 2021 & 2290000 & 112246 & 58748  & 55329  & 49580 \\
    R05000 & 秋田県 & 2021 & 945000  & 38992  & 21924  & 21448  & 8904  \\
    &        &      &          & $\vdots$  &        &        &       \\
    % R06000 & 山形県  & 2021 & 1055000  & 49164  & 26969  & 27233  & 11801  \\
    % R07000 & 福島県  & 2021 & 1812000  & 85322  & 46148  & 45647  & 14385  \\
    % R08000 & 茨城県  & 2021 & 2852000  & 135782 & 72465  & 71842  & 30147  \\
    % R09000 & 栃木県  & 2021 & 1921000  & 95315  & 51170  & 49674  & 20496  \\
    % R10000 & 群馬県  & 2021 & 1927000  & 94185  & 50841  & 48521  & 28772  \\
    % R11000 & 埼玉県  & 2021 & 7340000  & 363199 & 187395 & 163986 & 109500 \\
    % R12000 & 千葉県  & 2021 & 6275000  & 306105 & 158265 & 141358 & 106037 \\
    % R13000 & 東京都  & 2021 & 14010000 & 622820 & 311049 & 301712 & 676964 \\
    % R14000 & 神奈川県 & 2021 & 9236000  & 451098 & 226599 & 195931 & 171164 \\
    % R15000 & 新潟県  & 2021 & 2177000  & 103680 & 53720  & 51594  & 27546  \\
    % R16000 & 富山県  & 2021 & 1025000  & 47818  & 26146  & 26068  & 10857  \\
    % R17000 & 石川県  & 2021 & 1125000  & 56620  & 30336  & 29764  & 27627  \\
    % R18000 & 福井県  & 2021 & 760000   & 39236  & 21196  & 20701  & 10065  \\
    % R19000 & 山梨県  & 2021 & 805000   & 38572  & 20955  & 22717  & 16097  \\
    % R20000 & 長野県  & 2021 & 2033000  & 101932 & 55189  & 52632  & 17032  \\
    % R21000 & 岐阜県  & 2021 & 1961000  & 101805 & 54493  & 50563  & 20185  \\
    % R22000 & 静岡県  & 2021 & 3608000  & 183614 & 98192  & 91613  & 33778  \\
    % R23000 & 愛知県  & 2021 & 7517000  & 405839 & 209151 & 185920 & 176722 \\
    % R24000 & 三重県  & 2021 & 1756000  & 90040  & 47567  & 44229  & 14062  \\
    % R25000 & 滋賀県  & 2021 & 1411000  & 80289  & 41086  & 36673  & 31242  \\
    % R26000 & 京都府  & 2021 & 2561000  & 119892 & 65187  & 66457  & 143095 \\
    % R27000 & 大阪府  & 2021 & 8806000  & 422433 & 221610 & 207262 & 228194 \\
    % R28000 & 兵庫県  & 2021 & 5432000  & 278500 & 143075 & 128298 & 115536 \\
    % R29000 & 奈良県  & 2021 & 1315000  & 65989  & 35964  & 32530  & 20512  \\
    % R30000 & 和歌山県 & 2021 & 914000   & 43676  & 23677  & 23349  & 7891   \\
    % R31000 & 鳥取県  & 2021 & 549000   & 28027  & 14316  & 14321  & 6721   \\
    % R32000 & 島根県  & 2021 & 665000   & 33162  & 17040  & 17145  & 7263   \\
    % R33000 & 岡山県  & 2021 & 1876000  & 97981  & 50820  & 49501  & 39071  \\
    % R34000 & 広島県  & 2021 & 2780000  & 147671 & 75326  & 68044  & 55487  \\
    % R35000 & 山口県  & 2021 & 1328000  & 65000  & 33721  & 30983  & 18427  \\
    % R36000 & 徳島県  & 2021 & 712000   & 34181  & 17432  & 16965  & 11761  \\
    % R37000 & 香川県  & 2021 & 942000   & 49196  & 25629  & 24657  & 9118   \\
    % R38000 & 愛媛県  & 2021 & 1321000  & 66494  & 33330  & 31473  & 16321  \\
    % R39000 & 高知県  & 2021 & 684000   & 31226  & 16988  & 17139  & 9257   \\
    % R40000 & 福岡県  & 2021 & 5124000  & 279290 & 139657 & 123508 & 109860 \\
    % R41000 & 佐賀県  & 2021 & 806000   & 43903  & 23530  & 22422  & 7776   \\
    % R42000 & 長崎県  & 2021 & 1297000  & 68834  & 35782  & 34415  & 17083  \\
    % R43000 & 熊本県  & 2021 & 1728000  & 96415  & 48862  & 44284  & 24580  \\
    % R44000 & 大分県  & 2021 & 1114000  & 56464  & 29624  & 29300  & 15189  \\
    % R45000 & 宮崎県  & 2021 & 1061000  & 59639  & 30562  & 28856  & 9736   \\
    R46000 & 鹿児島県 & 2021 & 1576000  & 88636  & 45294  & 43029  & 15477  \\
    R47000 & 沖縄県  & 2021 & 1468000  & 101342 & 49716  & 43221  & 17882  \\
    R01000 & 北海道  & 2020 & 5224614  & 236396 & 123129 & 119773 & 79409 \\
           &        &      &          & $\vdots$  &        &        &       \\ \bottomrule
    \end{tabular}
    }
    \label{tab:population-table}
\end{table}

\section{射影(SELECT)}
最も単純なSQLはSELECT句とFROM句のみからなるものである. \strong{射影(projection)}とはFROM句で指定されたテーブルから,SELECT句で指定した特定の列のデータのみを抽出する操作である.

例題として用いるpopulationテーブルには,列として「地域コード」「都道府県」「調査年度」「総人口」「小学校児童数」「中学校生徒数」「高等学校生徒数」「大学学生数」があるが,ケースによって特定の列のデータのみ欲しい場合がある. そのようなケースで用いるのが射影である. 例えば,populationテーブルから「都道府県」「調査年度」「総人口」の列のデータのみを抽出する場合,SQL文はコード\ref{code:sql-select}となる.
\begin{figure}[tb]
    \begin{minted}[gobble=1,
        bgcolor=background,
        fontsize=\small,
        linenos=false,
        xleftmargin=0em]{sql}

    SELECT
        都道府県, 調査年度, 総人口
    FROM
        population;
    \end{minted}
    \captionsetup{name=コード}
    \caption{射影 - SELECT}
    \label{code:sql-select}
\end{figure}
SELECT句内で指定した「都道府県」「調査年度」「総人口」列のデータのみが表示された.

\begin{table}[tb]
    \centering
    \begin{tabular}{rrr}
    \toprule
    \textbf{都道府県} & \textbf{調査年度} & \textbf{総人口} \\ \midrule
    北海道           & 2021          & 5183000      \\
    青森県           & 2021          & 1221000      \\
    岩手県           & 2021          & 1196000      \\
    宮城県           & 2021          & 2290000      \\
    秋田県           & 2021          & 945000       \\
                  & $\vdots$      &              \\
    沖縄県           & 2020          & 1467480      \\ \bottomrule
    \end{tabular}
    \caption{}
    \label{tb:}
\end{table}

populationテーブルは合計で94行のレコードが格納されているため,表が縦に長くなってしまう.
SQL文の問い合わせで得られた結果のうち,先頭の$N$行だけを表示させるためには,コード\ref{code:sql-select2}のようにLIMIT句を使うとよい.
\begin{figure}[tb]
    \begin{minted}[gobble=1,
        bgcolor=background,
        fontsize=\small,
        linenos=false,
        xleftmargin=0em]{sql}

    SELECT
        都道府県, 調査年度, 総人口
    FROM
        population
    LIMIT 5;  -- コメント:先頭の5件のみ表示
    \end{minted}
    \captionsetup{name=コード}
    \caption{純なSELECT文}
    \label{code:sql-select2}
\end{figure}
\begin{table}[tb]
    \centering
    \begin{tabular}{rrr}
    \toprule
    \textbf{都道府県} & \textbf{調査年度} & \textbf{総人口} \\ \midrule
    北海道           & 2021          & 5183000      \\
    青森県           & 2021          & 1221000      \\
    岩手県           & 2021          & 1196000      \\
    宮城県           & 2021          & 2290000      \\
    秋田県           & 2021          & 945000       \\ \bottomrule
    \end{tabular}
    \caption{}
    \label{tb:}
\end{table}
コード\ref{code:sql-select2}のSQL文では表示する列を「都道府県」「調査年度」「総人口」に限定したが,テーブルがもつすべての列情報を表示させたい場合もあるだろう.
そのようなケースでは,コード\ref{code:sql-select3}のようにSELECT句に\strong{アスタリスク(*)}を使えばよい(アスタリスクを用いると,すべての列名を列挙するのと同等の結果が得られる).
\begin{figure}[tb]
    \begin{minted}[gobble=1,
        bgcolor=background,
        fontsize=\small,
        linenos=false,
        xleftmargin=0em]{sql}

    SELECT
        *
    FROM
        population
    LIMIT 5;
    \end{minted}
    \captionsetup{name=コード}
    \caption{単なSELECT文}
    \label{code:sql-select3}
\end{figure}
% Please add the following required packages to your document preamble:
% \usepackage{booktabs}
% \usepackage[table,xcdraw]{xcolor}
% Beamer presentation requires \usepackage{colortbl} instead of \usepackage[table,xcdraw]{xcolor}
\begin{table}[tb]
    \centering
    \scalebox{0.72}{
    \begin{tabular}{rrrrrrrr}
    \toprule
    \textbf{地域コード} & \textbf{都道府県} & \textbf{調査年度} & \textbf{総人口} & \textbf{小学校児童数} & \textbf{中学校生徒数} & \textbf{高等学校生徒数} & \textbf{大学学生数} \\ \midrule
    R01000 & 北海道 & 2021 & 5183000 & 231714 & 122742 & 115335 & 79729 \\
    R02000 & 青森県 & 2021 & 1221000 & 54460  & 29940  & 30543  & 15419 \\
    R03000 & 岩手県 & 2021 & 1196000 & 55597  & 30269  & 29980  & 11340 \\
    R04000 & 宮城県 & 2021 & 2290000 & 112246 & 58748  & 55329  & 49580 \\
    R05000 & 秋田県 & 2021 & 945000  & 38992  & 21924  & 21448  & 8904  \\ \bottomrule
    \end{tabular}
    }
    \caption{}
    \label{tb:}
\end{table}

\begin{tipbox}{読みやすいSQL}
SQL文では改行や余分な空白は無視される.
そのため,上記SQL文は\graybox{SELECT * FROM population LIMIT 10;}と解釈される.

 それでもわざわざ改行や余分な空白を入れたりしているのは,SQL文を読みやすくするためである.
複雑な問い合わせを行う場合,SQL文も複雑かつ長くなる.
そういったSQL文を読むのは苦痛であるしミスも見落としやすくなるので,できる限りSQL文を読みやすくしておくほうがよい.
\end{tipbox}


\section{選択(WHERE)}
\strong{選択}とは関係データベースから特定の条件を満たすレコードを抽出する操作である.
選択を行うにはWHERE句はを用いる.
WHERE句を使うことで,列の値と定数,あるいは列同士の値を比較して,FROM句で参照したテーブル中のデータを絞り込むことができる.

WHERE句内で使える代表的な比較演算子は,以下の通りである:
\begin{itemize}
\item $=$(等しい)
\item $!=$(等しくない)
\item $<$(より小さい)
\item $>$(より大きい)
\item $<=$(以下)
\item $>=$(以上)
\end{itemize}
例えば,populationテーブルから総人口数が750万以上のレコードを抽出するSQL文は以下となる.
\begin{figure}[tb]
    \begin{minted}[gobble=1,
        bgcolor=background,
        fontsize=\small,
        linenos=false,
        xleftmargin=0em]{sql}
    SELECT
        *
    FROM
        population
    WHERE
        総人口 >= 7500000;
    \end{minted}
    \captionsetup{name=コード}
    \caption{純なSELECT文}
    \label{code:sql-where}
\end{figure}
\begin{table}[tb]
    \centering
    \scalebox{0.72}{
    \begin{tabular}{rrrrrrrr}
    \toprule
    \textbf{地域コード} &
      \textbf{都道府県} &
      \textbf{調査年度} &
      \textbf{総人口} &
      \textbf{小学校児童数} &
      \textbf{中学校生徒数} &
      \textbf{高等学校生徒数} &
      \textbf{大学学生数} \\ \midrule
    R13000 & 東京都  & 2021 & 14010000 & 622820 & 311049 & 301712 & 676964 \\
    R14000 & 神奈川県 & 2021 & 9236000  & 451098 & 226599 & 195931 & 171164 \\
    R23000 & 愛知県  & 2021 & 7517000  & 405839 & 209151 & 185920 & 176722 \\
    R27000 & 大阪府  & 2021 & 8806000  & 422433 & 221610 & 207262 & 228194 \\
    R13000 & 東京都  & 2020 & 14047594 & 619291 & 304405 & 306302 & 673683 \\
     & & & $\vdots$ & & & & \\ \bottomrule
    \end{tabular}
    }
    \caption{}
    \label{tb:}
\end{table}

比較したい列のデータ型が文字列の場合は比較したい文字列を\strong{ダブルクォーテーション('')}もしくは\strong{シングルクォーテーション(')}で囲う必要がある.
以下は,populationテーブルから都道府県名が「京都府」のレコードを抽出するSQL文の例である.
\begin{figure}[tb]
    \begin{minted}[gobble=1,
        bgcolor=background,
        fontsize=\small,
        linenos=false,
        xleftmargin=0em]{sql}

    SELECT
        *
    FROM
        population
    WHERE
        都道府県 = "京都府";
    \end{minted}
    \captionsetup{name=コード}
    \caption{純なSELECT文}
    \label{code:sql-where2}
\end{figure}
\begin{table}[tb]
    \centering
    \scalebox{0.73}{
    \begin{tabular}{rrrrrrrr}
    \toprule
    \textbf{地域コード} & \textbf{都道府県} & \textbf{調査年度} & \textbf{総人口} & \textbf{小学校児童数} & \textbf{中学校生徒数} & \textbf{高等学校生徒数} & \textbf{大学学生数} \\ \midrule
    R26000 & 京都府 & 2021 & 2561000  & 119892 & 65187  & 66457  & 143095 \\
    R26000 & 京都府 & 2020 & 2578087  & 121712 & 65443  & 67847  & 141870 \\ \bottomrule
    \end{tabular}
    }
    \caption{}
    \label{tb:}
\end{table}

文字列の条件指定においては,列データに特定の文字列を含むレコードを抽出したいケースもある.
そのようなケースではLIKE句を用いる.
以下の例のように,WHERE句内に\graybox{LIKE \%部分文字列\%}といった文を書くと,指定した部分文字列を列データに含むレコードに絞り込むことができる.
なお\%(パーセント)は0文字以上の任意の文字列を意味する.
以下は,populationテーブルから都道府県名に「京都」の文字を含むレコードのみを抽出するSQL文の例である.
\begin{figure}[tb]
    \begin{minted}[gobble=1,
        bgcolor=background,
        fontsize=\small,
        linenos=false]{sql}
    SELECT
        *
    FROM
        population
    WHERE
        都道府県 LIKE "%京都%";
    \end{minted}
    \captionsetup{name=コード}
    \caption{純なSELECT文}
    \label{code:sql-where3}
\end{figure}
\begin{table}[tb]
    \centering
    \scalebox{0.73}{
    \begin{tabular}{rrrrrrrr}
    \toprule
    \textbf{地域コード} & \textbf{都道府県} & \textbf{調査年度} & \textbf{総人口} & \textbf{小学校児童数} & \textbf{中学校生徒数} & \textbf{高等学校生徒数} & \textbf{大学学生数} \\ \midrule
    R13000 & 東京都 & 2021 & 14010000 & 622820 & 311049 & 301712 & 676964 \\
    R26000 & 京都府 & 2021 & 2561000  & 119892 & 65187  & 66457  & 143095 \\
    R13000 & 東京都 & 2020 & 14047594 & 619291 & 304405 & 306302 & 673683 \\
    R26000 & 京都府 & 2020 & 2578087  & 121712 & 65443  & 67847  & 141870 \\ \bottomrule
    \end{tabular}
    }
    \caption{}
    \label{tb:}
\end{table}
「京都」の前後にパーセント記号をつけることによって,都道府県名が「〜京都」もしくは「京都〜」のパターンにマッチするレコードのみに絞り込んでいる.
もしLIKE句の条件を\graybox{\%京都\%}ではなく\graybox{\%京都}にした場合,京都府のレコードはマッチしなくなる.

\begin{notebox}{SQLの処理順序}
    SELECT文による問い合わせが行われたとき,関係データベース管理システムは以下のステップで処理を行う.
    \begin{enumerate}
    \item FROM句で指定した表を参照
    \item 表中の各レコードがWHERE句で指定された条件を満たしているかを確認
    \item ステップ2で条件を満たしていると判定された行のみ,その行にあるSELECT句で指定した列のデータを表示
    \end{enumerate}
\end{notebox}

WHERE句では条件を複数指定することもできる. 条件は論理演算子である\graybox{AND}もしくは\graybox{OR}で結合することができる.
\begin{itemize}
\item \graybox{条件1 AND 条件2}と指定すれば,条件1と条件2をともに満たすレコード
\item \graybox{条件1 OR 条件2}と指定すれば,条件1もしくは条件2のいずれかを満たすレコード
\end{itemize}
を抽出することができる.

以下は,populationテーブルから総人口が100万人以上でかつ大学生数が高校生数よりも多い都道府県を抽出するSQL文の例である
\begin{figure}[tb]
    \begin{minted}[gobble=1,
        bgcolor=background,
        fontsize=\small,
        linenos=false]{sql}
    SELECT
        都道府県
    FROM
        population
    WHERE
        (総人口 >= 1000000)
        AND (大学学生数 > 高等学校生徒数);
    \end{minted}
    \captionsetup{name=コード}
    \caption{純なSELECT文}
    \label{code:sql-where3}
\end{figure}
\begin{table}[tb]
    \centering
    \begin{tabular}{r}
    \toprule
    \textbf{都道府県} \\ \midrule
    東京都           \\
    京都府           \\
    大阪府           \\
    東京都           \\
    京都府           \\
    大阪府           \\ \bottomrule
    \end{tabular}
    \caption{}
    \label{tb:}
\end{table}

\begin{tipbox}{条件の明確化}
    条件の前後に丸括弧をつけることで,条件の記述範囲を明確にすることができる.
    条件の範囲やAND/ORのかかる順序をわかりやすくするためにも,3つ以上の条件を組み合わせるような場合には条件の前後に丸括弧をつけることをオススメする.
\end{tipbox}

上記結果には重複する結果が含まれているが,通常SQLは重複した結果があってもそのまま出力される.
行の重複を除いた結果を出力したい場合,SELECTの直後に\graybox{DISTINCT}を指定する.

上記の例においては,以下のようなSQL文を発行すると重複のない結果が得られる.
\begin{figure}[tb]
    \begin{minted}[gobble=1,
        bgcolor=background,
        fontsize=\small,
        linenos=false]{sql}
    SELECT DISTINCT
        都道府県
    FROM
        population
    WHERE
        (総人口 >= 1000000)
        AND (大学学生数 > 高等学校生徒数);
    \end{minted}
    \captionsetup{name=コード}
    \caption{純なSELECT文}
    \label{code:sql-where3}
\end{figure}
\begin{table}[tb]
    \centering
    \begin{tabular}{r}
    \toprule
    \textbf{都道府県} \\ \midrule
    東京都           \\
    京都府           \\
    大阪府           \\ \bottomrule
    \end{tabular}
    \caption{}
    \label{tb:}
\end{table}

\section{整列(ORDER BY)}
データ分析では,大きいもの(小さいもの)順にデータを\strong{整列(ソート)}させたいケースが多々ある. そのようなケースで使用するのが\graybox{ORDER BY}句である.

以下のように\graybox{ORDER BY}の後に列名を指定することで,SQL文で抽出したレコードを指定した列の値の小さいもの順(昇順) に並び替えることができる.
\begin{figure}[tb]
    \begin{minted}[gobble=1,
        bgcolor=background,
        fontsize=\small,
        linenos=false]{sql}
    SELECT
        *
    FROM
        population
    ORDER BY
        総人口
    LIMIT 5; --- 先頭の5件のみ表示
    \end{minted}
    \captionsetup{name=コード}
    \caption{純なSELECT文}
    \label{code:sql-where3}
\end{figure}
\begin{table}[tb]
    \centering
    \scalebox{0.72}{
    \begin{tabular}{rrrrrrrr}
    \toprule
    \textbf{地域コード} &
      \textbf{都道府県} &
      \textbf{調査年度} &
      \textbf{総人口} &
      \textbf{小学校児童数} &
      \textbf{中学校生徒数} &
      \textbf{高等学校生徒数} &
      \textbf{大学学生数} \\ \midrule
    R31000 & 鳥取県 & 2021 & 549000 & 28027 & 14316 & 14321 & 6721 \\
    R31000 & 鳥取県 & 2020 & 553407 & 28238 & 14522 & 14572 & 6736 \\
    R32000 & 島根県 & 2021 & 665000 & 33162 & 17040 & 17145 & 7263 \\
    R32000 & 島根県 & 2020 & 671126 & 33921 & 17119 & 17707 & 7098 \\
    R39000 & 高知県 & 2021 & 684000 & 31226 & 16988 & 17139 & 9257 \\
     & & & $\vdots$ & & & & \\ \bottomrule
    \end{tabular}
    }
    \caption{}
    \label{tb:}
\end{table}

\graybox{ORDER BY}はデフォルトは小さいもの順(昇順,in ascending order)でレコードをソートする.
大きいもの順(降順, in descending order)でソートしたい場合は,\graybox{ORDER BY}で列名を指定する際,列名の後に\graybox{DESC}キーワードを付ける.
\begin{figure}[tb]
    \begin{minted}[gobble=1,
        bgcolor=background,
        fontsize=\small,
        linenos=false]{sql}
    SELECT
        *
    FROM
        population
    ORDER BY
        総人口 DESC --- DESCを付けることで総人口の降順で結果を並び替える
    LIMIT 5;
    \end{minted}
    \captionsetup{name=コード}
    \caption{純なSELECT文}
    \label{code:sql-where3}
\end{figure}
\begin{table}[tb]
    \centering
    \scalebox{0.72}{
    \begin{tabular}{rrrrrrrr}
    \toprule
    \textbf{地域コード} &
      \textbf{都道府県} &
      \textbf{調査年度} &
      \textbf{総人口} &
      \textbf{小学校児童数} &
      \textbf{中学校生徒数} &
      \textbf{高等学校生徒数} &
      \textbf{大学学生数} \\ \midrule
      R13000 & 東京都  & 2020 & 14047594 & 619291 & 304405 & 306302 & 673683 \\
      R13000 & 東京都  & 2021 & 14010000 & 622820 & 311049 & 301712 & 676964 \\
      R14000 & 神奈川県 & 2020 & 9237337  & 454751 & 224709 & 200230 & 174710 \\
      R14000 & 神奈川県 & 2021 & 9236000  & 451098 & 226599 & 195931 & 171164 \\
      R27000 & 大阪府  & 2020 & 8837685  & 427884 & 220342 & 214115 & 226452 \\
     & & & $\vdots$ & & & & \\ \bottomrule
    \end{tabular}
    }
    \caption{}
    \label{tb:}
\end{table}

\section{集約関数}
合計値や平均値の計算など,データの集計はデータ集合の特徴を知る上での基礎となる.
関係データベースから抽出したレコード集合に対して集計処理を行いたい場合は\strong{集約関数(aggregate functions)}を用いる.
SQLに実装されている代表的な集約関数は以下の通りである.

\begin{itemize}
\item \graybox{SUM}: 合計値の計算
\item \graybox{MAX}: 最大値の計算
\item \graybox{MIN}: 最小値の計算
\item \graybox{AVG}: 平均値の計算
\item \graybox{COUNT}: 行数のカウント
\end{itemize}
関数の引数には計算に用いたい列名などを指定する.
(次節で解説する\graybox{GROUP BY}句を用いない場合)集計関数は,\strong{\graybox{WHERE}句までで絞り込まれたレコード全体を1つのグループと見なして}集計処理を行う.

例えば,\graybox{population}テーブルから調査年度が2021年度のレコードだけに限定して,各都道府県の総人口の合計値を計算するSQL文は以下となる.


\graybox{SELECT}句に複数集約関数を指定することで,複数の集計処理を同時に行うこともできる.
以下は,\graybox{population}テーブルを調査年度が2021年度のレコードを用いて,各都道府県の総人口の「合計値」「平均値」「最大値と最小値の差」を計算するSQL文である.

\begin{notebox}{様々な演算子}
\graybox{SELECT}句や\graybox{WHERE}句の中では,四則演算も行うことができる.
加算(+),減算(-),乗算(*),除算(/)といった四則演算のための演算子以外にも,余りを求めるための剰余演算子(\%),絶対値を求めるABS関数といった様々な算術演算ツールが用意されている.
気になった演算があればマニュアルを調べてみよう.
\end{notebox}

行数(レコード数)をカウントする\graybox{COUNT}関数は,しばしば引数にすべての列を意味するアスタリスク(*)が用いられる. 以下は,\graybox{population}テーブルに格納されたレコード数を調べるSQL文である.

上記のSQL文は\graybox{SELECT * FROM population;}の実行結果の行数を数えていると考えればよい.
以下のSQL文は結果だけ見ると上記のSQL文と同じになるが,処理の流れとしては\graybox{SELECT 都道府県 FROM population;}の実行結果の行数を数えていることになる.

なお,上記の結果には2021年度と2020年度の結果が含まれているため,同じ都道府県名が2回数えられてしまっている.
都道府県名の重複を除いて行数をカウントしたい場合は,以下のSQL文のように\graybox{DISTINCT}を使う.

\section{グループ化による集約演算(GROUP BY)}
前節までに解説した集約のためのSQL文は,\graybox{WHERE}句で絞り込まれたレコード全体に対して集計操作を行うものであった.
しかし実際にデータ分析を行う場合,レコード全体での集計にとどまらず,ある基準でまとめられたグループごとに集計を行うことも少なくない.
\graybox{GROUP BY}句はグループごとに集約演算を行うための機能である.

「\graybox{GROUP BY}句 + 列名」の形式で指定をすると,指定された列名について同じ値をもつレコードが1つのグループにまとめられた\strong{グループ表}が(ユーザには見えない形で)一時的に作成される.
例えば,\graybox{population}テーブルに対して問い合わせを行うSQL文内で「\graybox{GROUP BY} 地域コード」と書くと,以下のようなイメージのグループ表が一時的に作成される(点線がグループを表す).

\graybox{GROUP BY}句を用いたクエリを用いると,まとめられたグループのそれぞれに対して\graybox{SELECT}句で指定された集約関数が適用される.

以下は,\graybox{population}テーブルを用いて,都道府県ごとに2021年と2020年の総人口の平均値を計算するSQL文である.

\graybox{GROUP BY}句は\graybox{WHERE}句と組み合わせて使うこともできる. \graybox{WHERE}句を使うことで,ある条件で絞り込んだレコード集合に対して\graybox{GROUP BY}を適用することができる.

以下は,\graybox{population}テーブルの中で総人口が500万を超えるレコードに限定して,都道府県ごとに2021年と2020年の総人口の平均値を計算するSQL文である

グループごとに集計した結果にチェックを行い,指定した条件を満たした結果を抽出したいケースもある.
そのようなケースでは\graybox{HAVING}句を用いる.
\graybox{HAVING}句は\graybox{WHERE}句と同じ形式で条件を指定するが,\graybox{GROUP BY}句の後に書くことに注意しよう(\graybox{WHERE}句は\graybox{GROUP BY}句の前).

以下は,\graybox{population}テーブルのレコードについて,都道府県ごとに2021年と2020年の「大学学生数」の平均値を計算し,平均大学学生数が10万を超えたものについて,都道府県名,平均総人口,平均大学学生数を表示するSQL文である.
\graybox{WHERE}句を用いた場合と\graybox{HAVING}句を用いた場合で挙動が異なることを意識しよう

\begin{notebox}{SQLの処理順序(再び)}
本章では\graybox{SELECT}句,\graybox{FROM}句,\graybox{WHERE}句,\graybox{ORDER BY}句,\graybox{GROUP BY}句,\graybox{HAVING}句が登場したが,これらが用いられたSELECT文による問い合わせが行われたとき,どのような順序で処理が行われているかを意識しよう.
関係データベース管理システムは以下のステップで処理を行う.
\begin{enumerate}
\item \graybox{FROM}句で指定した表を参照
\item \graybox{WHERE}句があれば\graybox{WHERE}句で指定された条件を満たすレコードを選択.なければ\graybox{FROM}句で指定した表中の全レコードを選択.
\item (\graybox{GROUP BY}句があれば)ステップ2で選択されたレコードをグループ化する
\item (\graybox{HAVING}句があれば)ステップ3でまとめられたグループに対する条件付けを行う
\item (\graybox{ORDER BY}句があれば)指定された基準に基づきレコードをソートする
\item 条件を満たしたものについて,\graybox{SELECT}句で指定された値を表示.
\end{enumerate}
\end{notebox}


\backmatter
\bibliography{reference}
\bibliographystyle{tipsj}

\end{document}
